\documentclass{article}

\usepackage[margin=1in]{geometry}

\usepackage{amssymb,amsmath}
\usepackage{amsthm}

\begin{document}
  Two of the basic processing modes in \LaTeX{} are \emph{text mode} and
  \emph{math mode}. Text mode is processing the paragraph you are now reading;
  it reads the characters from the input file, generates a glyph (symbol) for
  each one and builds words, lines and paragraphs from the resulting sequence.

  Math mode is used to typeset mathematics; the input is processed differently
  than text mode. Three key differences:
  \begin{itemize}
    \item Spaces are used for delimiting (ending) commands, but do not generate
    a space in the rendered output
    \item Text is italicized, except for certain well-known math terms
    (\emph{e.g.}, sin)
    \item Certain operators, like \_ and \^{} have special meaning
  \end{itemize}
  \vskip6pt

  There are several different varieties of math modes; I will show you three of
  the more common ones here.\\

  The first variety is \emph{inline math mode}. This places a small piece of
  typeset math within a text paragraph. For example, {\tt \$0\textbackslash
  le x\textbackslash le 10\$} renders as $0\le x\le 10$ within this paragraph.

  Inline math mode begins and ends with the {\tt \$} symbol. If you omit either,
  the \LaTeX{} processor will give you an error.

  Inline math is good for smaller math items, but note that some math doesn't
  render very well inline. For example, the summation formula {\tt \$%
  \textbackslash sum\_\{i=0\}\^{}n i=\textbackslash frac\{n(n+1)\}\{2\}\$}
  renders as $\sum_{i=0}^n i=\frac{n(n+1)}{2}$ in inline mode.\\

  The next variety is \emph{display math mode}. The input is the same as inline
  mode, but the delimiters are {\tt \textbackslash [} and {\tt \textbackslash
  ]}. Often, these are on their own line to make reading the input file a bit
  easier.

  Display mode separates the math typesetting apart from the text surrounding
  it, effectively making the math its own paragraph (which, behind the scenes,
  is exactly what happens). The main effects, other than creating its own
  paragraph, are that larger symbols are used for things like summations and
  integrals, and sufficient vertical space is used to properly render all of the
  input.\\

  The above summation in display mode renders as:
  \[
    \sum_{i=0}^n i=\frac{n(n+1)}{2}
  \]
  \vskip6pt

  The third variety is a variation of display mode. The \emph{equation}
  environment allows equations to be numbered for reference elsewhere in the
  document. The environment begins with {\tt \textbackslash begin\{equation\}}
  and ends with {\tt \textbackslash end\{equation\}}. Note that all environments
  begin and end with {\tt \textbackslash begin\{\ldots\}} and {\tt
  \textbackslash end\{\ldots\}}.

  The summation formula renders in the {\tt equation} environment as:

  \begin{equation}
    \sum_{i=0}^n i=\frac{n(n+1)}{2}\label{eq:sum-of-integers}
\end{equation}

Note the only difference is the $(1)$ at the right edge. This can be used to
refer back to the equation elsewhere in the document.\\

Although you can manually insert the equation number, the reference feature of
\LaTeX{} is commonly used to handle numbered equations, algorithms, figures,
\emph{etc}.
Note that in the source document, I added {\tt \textbackslash label\{eq:sum-of-%
integers\}} after the formula. This creates a \emph{label} that can later be
used.\\

To refer to an equation, use the {\tt \textbackslash ref\{\ldots\}} command.
To refer to the summation, {\tt \textbackslash ref\{eq:sum-of-integers\}} will
produce the equation number. However, it only generates the number. Normally,
you would use something like {\tt Equation\textasciitilde \textbackslash ref%
\{eq:sum-of-integers\}} to get ``Equation~\ref{eq:sum-of-integers}''.\\

Important note: You typically have to run the rendering command twice --- and
thre times for complex documents --- to get reference numbering to render
properly.
\end{document}
